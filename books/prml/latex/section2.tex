\documentclass[a4j,twocolumn, uplatex]{jsarticle}

\usepackage{bm}

\begin{document}

\section{頻度主義、ベイズ主義}

頻度主義(frequensitst)では、まず確率モデルを仮定して、得られたデータから確率モデルのパラメーターを予測する。
例えば、「この実験データはガウス分布に従って得られているので、真の平均値を推測する」というのは頻度主義である。
つまり、最尤法でパラメータを求めるのは頻度主義に立脚した手法である。

ベイズ主義では、このデータが得られる尤もらしい確率分布を予測する。

どちらが正しい、ということはない。
そのば場その場で適切な主義を主張できるのが重要である。


\end{document}
