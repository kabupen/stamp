\documentclass[a4j,twocolumn, uplatex]{jsarticle}

\usepackage{bm}

\begin{document}

\section{決定理論}
%2つの確率変数$X$, $Y$を考える。$X$かつ$Y$が起きる確率は同時確率(joint probability)と呼ばれ、
%\begin{equation}
%  p(X,Y)
%\end{equation}
%で表される。
%周辺確率(marginal probability)
決定理論とは、パターン認識のように確率論をベースにして意思決定を行うための枠組みである。
同時確率からベイズの定理を経由して、確率から最適な決定を導き出すことができる。

まず訓練データを用いて、入力変数$\bm{x}$と対応する目標変数$\bm{t}$(回帰問題であれば連続変数、分類問題であればクラスラベルの離散変数)
から同時確率$p(\bm{x},\bm{t})$を求める。
以下ではクラス分類モデルに対しての決定理論を構築する。

ベイズの定理から、「ある入力$\bm{x}$に対して、クラス$C_k$である確率」を表現することができ、
\begin{equation}
  p(C_k|\bm{x}) = \frac{p(\bm{x}|C_k)p(C_k)}{p(\bm{x})}
\end{equation}
これらの情報は全て同時確率から求めることができる。
\begin{eqnarray}
  && p(\bm{x}|C_k)p(C_k) = p(\bm{x}, C_k) \\
  && p(\bm{x}) =  \int p(\bm{x}, C_k) dC_k
\end{eqnarray}



\end{document}
