\documentclass[a4j,twocolumn, uplatex]{jsarticle}

\usepackage{bm}

\begin{document}

\section{確率についての立場}

確率について少し調べるとよく、「頻度論主義(frequentist)」と「ベイズ主義(beysian)」が歴史的にはあり...という話に行き着く。
結論としてはどちらの立場(主義)でも間違ったことはしていないので、最も賢い取り組み方は
問題に対してどちらの立場で物事を考えれば最も簡単に解決に迎えるかを考えることだと思う。

\subsection{頻度論}

いたってシンプルに、何度も独立試行を行って確率問題に取り組む方法。
サイコロを投げて1が出る確率を求めたいときには、何度もサイコロを振って1の目が出た回数$n$を
その振った総数$N$で割る。これは頻度論主義に立脚した考えであり、「1の目が出る」という確率は
真に決まっていて、それを何度も実験を繰り返して推測するという手法である。つまり、頻度論主義とは
\begin{equation}
  \lim_{N\to\infty} \frac{n}{N}
\end{equation}
において、$N$を無限大に持っていく(何度も試行を繰り返す)ということを表現した考え方である。
また、「1の目が出る確率」は真に決まっているということも、頻度論主義の特徴である。

\subsection{ベイズ主義}


\section{決定理論}
%2つの確率変数$X$, $Y$を考える。$X$かつ$Y$が起きる確率は同時確率(joint probability)と呼ばれ、
%\begin{equation}
%  p(X,Y)
%\end{equation}
%で表される。
%周辺確率(marginal probability)
決定理論とは、パターン認識のように確率論をベースにして意思決定を行うための枠組みである。
同時確率からベイズの定理を経由して、確率から最適な決定を導き出すことができる。

まず訓練データを用いて、入力変数$\bm{x}$と対応する目標変数$\bm{t}$(回帰問題であれば連続変数、分類問題であればクラスラベルの離散変数)
から同時確率$p(\bm{x},\bm{t})$を求める。
以下ではクラス分類モデルに対しての決定理論を構築する。

ベイズの定理から、「ある入力$\bm{x}$に対して、クラス$C_k$である確率」を表現することができ、
\begin{equation}
  p(C_k|\bm{x}) = \frac{p(\bm{x}|C_k)p(C_k)}{p(\bm{x})}
\end{equation}
これらの情報は全て同時確率から求めることができる。
\begin{eqnarray}
  && p(\bm{x}|C_k)p(C_k) = p(\bm{x}, C_k) \\
  && p(\bm{x}) =  \int p(\bm{x}, C_k) dC_k
\end{eqnarray}

\section{誤り確率}





\end{document}
