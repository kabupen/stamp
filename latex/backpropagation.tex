\documentclass[a4j, uplatex]{jsarticle}

\usepackage{bm}

\begin{document}

\section{線形について}

以下を満たす関数を線形写像という。

\begin{eqnarray}
    f(x+y) &=& f(x) + f(y) \\
    f(ax) &=& af(x)
\end{eqnarray}


とくに $f(ax)=af(x)$ について。関数の外に$a$を出せる場合、線形であると考えれることは盲点である。

\section{行列}

\subsection{正方行列}

$n\times n$の正方形の形をした行列のこと

\subsection{対角行列}

正方行列 $A = a_{ij}$ において、対角成分 ($a_{ii}$) 以外がすべて 0 であるような行列を対角行列とよぶ


\subsection{実対称行列}

$m \times n $行列 $A$ に対して、$a_{ij} = a_{ji}$ が成り立つ場合、対称行列とよぶ


\subsection{反対称行列、交代行列}

$A^T = -A$


\subsection{転置行列}

$(m, n)$行列の縦横を逆にした $(n,m)$行列を $A$の転置行列という


\subsection{複素共役行列}

行列 $A=a_{ij}$ の各成分を、共役複素数で置き換えた行列 $\bar{A} = (\bar{a}_{ij})$ を複素共役行列という。


\subsection{直行 (orthogonal)}

$$
\mathrm{U^TU = I}
$$

$$
U^TUU^{-1} = U^{-1} \\
U^T = U^{-1}
$$


\subsection{正則行列}

$n$次行列$A$に対し、$XA=AX=E$ となる行列 $X$ (逆行列) が存在するとき、$A$を正則行列という。
$A$が正則ならば逆行列は一つしか無い。

\subsection{随伴行列、エルミート行列}

複素共役を取って転置した$\bar{A}^T$ を$A$の随伴行列と呼ぶ。
特に正方行列が $A=A^\ast$を満たす場合、Aをエルミート行列と呼ぶ。
特に実エルミート行列を実対称行列と呼ぶ。

$$
A = \overline{A}^T = A^\ast
$$


\subsection{ユニタリ行列}

正方行列$A$が $A^\ast A=E$をいたす場合、$A$をユニタリ行列と呼ぶ。特に実ユニタリ行列を直交行列と呼ぶ。




\section{誤差逆伝播法}

誤差関数は

\begin{equation}
    E_n = \frac{1}{2} \sum_{k=1}^{m} (y_k(\bm{x}) - d_j)^2
\end{equation}

ここで$m$は出力層のノード数(出力値の次元数)を表す。
ネットワークの訓練は、この誤差関数が最小値(極小)を持つように重み係数を設定することである。
そのために勾配降下法では
\begin{equation}
    w \leftarrow w - \eta \nabla E 
\end{equation}
として、現時点での誤差関数の勾配を使って重みの更新を行う。

第$\ell$層の重み$w_{ji}^\ell$についての勾配を計算する。
誤差関数は
\begin{eqnarray}
    E_n 
    &=& \frac{1}{2} \sum_{k=1}^{m} \left( f \left( w_{kj}^{(\ell)}z_j^{(\ell)} \right) - d_j \right)^2 \\
    &=& \frac{1}{2} \sum_{k=1}^{m} \left( f \left( w_{kj}^{(\ell-1)} f \left( w_{ji}^{(\ell-1)}z_i^{(\ell-2)} \right)\right) - d_j \right)^2 \\
\end{eqnarray}




\end{document}
